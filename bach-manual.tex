\documentclass[11pt, Palatino]{article}

\usepackage{amsmath}
\usepackage{amssymb}
\usepackage{graphicx,color}
\usepackage{bm}
\usepackage{amsthm}
\usepackage{amsfonts}
\usepackage{enumerate}
%\usepackage{natbib}
\usepackage{url}
\usepackage{setspace}
\usepackage{lineno}
\usepackage{multirow}

%\onehalfspacing
\addtolength{\oddsidemargin}{-.5in}%
\addtolength{\evensidemargin}{-.5in}%
\addtolength{\textwidth}{1in}%
\addtolength{\textheight}{1.3in}%
\addtolength{\topmargin}{-.8in}%
 

\def\bach{{\sc Bach}~}
\def\tcr{\textcolor{red}}

\begin{document}

\begin{center}
{\huge User Manual for \bach} \\
\vspace{0.5cm}
{\large Quan Zhou (zhouquan.stat@gmail.com) and Yongtao Guan} \\
\vspace{0.2cm}
{\large Last update: 7 Dec 2015} \\
\end{center}

\bach implements Bausch's algorithm to evaluate the distribution function of a weighted sum of $\chi_1^2$ random variables~\cite{bausch2013efficient}; one important application is to compute the p-values associated with the Bayes factors~\cite{quan.guan.2015}.  There are two ways to run \bach: simple or in batch. 
Note in the following example, we use ``./bach" as the program name; in the bach folder  there are ``bach-mac" (bach for Mac OS X) and ``bach-linux" (bach for Linux). You may either rename the executable or modifiy the command lines below.  
 
\section{Simple run}
Denoted by $\chi_1^2$ a chi-squared random variable (r.v.) with $1$ degree of freedom (d.f).  Let
$Y =  X_1 + 0.8 X_2 + 0.6 X_3 + 0.4 X_4,$
where $X_i\sim \chi_1^2$ independently. 
To compute $P(Y > 10),$ the command line is: 
\begin{verbatim}
	./bach 1 0.8 0.6 0.4 10
\end{verbatim}
A p-value will be printed to the screen:
\begin{verbatim}
	0.00914343 
\end{verbatim} 
\textbf{Rules of input:} Numbers are separated by space; the last number is the statistic and the other numbers are coefficients. 

\medskip
Suppose we also want to compute $P(Y>20)$. Of course we may repeat the above command line, substituting $10$ with $20.$ 
But a more efficient way is to compute $P(Y>20)$ and $P(Y>10)$ simultaneously because we can reuse the distribution functions.  
The command line is: 
\begin{verbatim}
	./bach 1 0.8 0.6 0.4 10,20
\end{verbatim}
Two p-values will be printed to the screen:
\begin{verbatim}
	0.00914343  4.31004e-5
\end{verbatim} 
\textbf{Rules of input:} If we want to compute multiple p-values for the same coefficients, just append more test statistics in the end using `,' as the delimiter. 
 
\section{Batch run}
Suppose $Z =  5 X_5 + 0.2 X_6,$ where $X_i \sim \chi_1^2$. 
If we want to compute $P(Z>34)$, $P(Y > 10)$ and $P(Y>20)$ simultaneously in batch mode, we can create the following file (named ``text.input"): 
\begin{verbatim}
	1 0.8 0.6 0.4 10,20
	5 0.2 34
\end{verbatim}
Each row in the input file is what you would type using command line save the ``./bach". To run \bach in batch mode: 
\begin{verbatim}
	./bach -i test.input
\end{verbatim}
Three p-values will then be printed on the screen:
\begin{verbatim}
	0.00914343 4.31004e-5
	0.00932733
\end{verbatim}
Within each row, multiple p-values are delimited by a tab ({`\textbackslash{t}');  each row in the output corresponds to the same row in the input file.   
  
\section{Output file}  
When `-o' argument is invoked for batch mode, an output file will be generated. 
\begin{verbatim}
	./bach -i test.input -o test.output
\end{verbatim}
This command will produce a `test.output' which reads
\begin{center}
{
\begin{tabular}{c c c c}
p-value &	error-bound & coeffcients & statistic \\
\hline
0.00914343	& 	3.02826e-15 & 0.4 0.6 0.8 1 & 10 \\
4.31004e-05 &	5.37358e-11 & 0.4 0.6 0.8 1 & 20 \\
0.00932733	& 3.55176e-08 & 0.2  5 & 34\\
\end{tabular} 
}
\end{center} 
The output file contains 4 columns: (1) p-value; (2) the upper bound of the absolute error; (3) coefficients of the $\chi_1^2$ variables; (4) the value of the statistic. 
The delimiter between fields is the tab, and within the field of ``coefficients" the delimiter is the blank space.  
 
\section{Options}
\begin{flushleft}
\begin{itemize}
\item -i [string=]: the input filename. This is required for batch mode. 
\item -o [string=]: the output filename for the batch mode. 
\item -c [int=6]: the precision of all output (number of significant digits, default is 6).
\item -h : stop the program and print the HELP.
\end{itemize}  
\medskip
There are additional options to control error bounds, some of which concern GMP library used in our implementation; one may send an email to inquire if on the off chance one is interested.  
\end{flushleft}


%\bibliographystyle{unsrt}
%\bibliography{/Users/yongtaog/Dropbox/Shared/quan/chi_square_code/doc/bf_ref}
\begin{thebibliography}{1}

\bibitem{bausch2013efficient}
Johannes Bausch.
\newblock On the efficient calculation of a linear combination of chi-square
  random variables with an application in counting string vacua.
\newblock {\em Journal of Physics A: Mathematical and Theoretical},
  46(50):505202, 2013.

\bibitem{quan.guan.2015}
Quan Zhou and Yongtao Guan.
\newblock On the null distribution of Bayes factors in linear regressin.
\newblock {\em JASA (Theory and Methods)}, 113(523): 1362 -- 1371, 2018.  

\end{thebibliography}


\end{document}

 



 